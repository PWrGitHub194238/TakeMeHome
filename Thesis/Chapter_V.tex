\chapter{Zakończenie}
\thispagestyle{chapterBeginStyle}





Celem powyższej pracy było nie tyle co przedstawić szeroki wachlarz algorytmów wyspecjalizowanych w odnajdywaniu najkrótszych ścieżek, co na drodze eksperymentów stwierdzić, które z nich dobrze sprawdzają się w swojej roli, które są warte polecenia dla konkretnych klas problemów, a których użycie należałoby odradzić.
Po przeprowadzeniu szczegółowej analizy dla każdego z takich algorytmów oraz wykonaniu serii eksperymentów można dojść do wniosku, że dla prawdziwych sieci drogowych godnymi polecenia są dwa z nich, zwłaszcza ze względu na prostotę implementacji: algorytm Dial oraz inna modyfikacja algorytmu Dijkstry oparta o kubełki wielopoziomowe (których stopień skomplikowania jest co prawda dalece większy, lecz daje on zdecydowanie najlepsze wyniki).
Szereg dodatkowych eksperymentów przeprowadzonych dla zmodyfikowanych danych wejściowych (poprzez zwiększanie kosztów wszystkich ścieżek w grafie) także okazał się wskazywać na przewagę tych algorytmów nad innymi, co tylko potwierdza wcześniejszą rekomendację algorytmu \textsc{DKD} jako najszybszego dla dużego rozmiaru sieci drogowych.
Dodatkowo także pokazaliśmy zalety posiadane przez algorytmy zbudowane w oparciu o kubełki pozycyjne (ang. \textit{RadixHeap}) oraz kopce Fibonacciego.
Ze względu na małą wrażliwość wobec powiększania kosztów w sieci algorytmy te także z powodzeniem znajdą zastosowanie w wyszukiwaniu najkrótszych ścieżek, z tym że implementacja algorytmu Dijkstry w oparciu o kopiec Fibonacciego jest zdecydowanie bardziej skomplikowana niż zaprogramowanie drugiego z rozwiązań.
Na sam koniec należy jeszcze zwrócić uwagę, że każdy z przedstawionych algorytmów zależał od liczby krawędzi w grafie, co przy ich liczbie rzędu $m = O \left( n^2 \right)$ potrafi uczynić bezcelową analizę najgorszego przypadku większości z omawianych rozwiązań. W takich przypadkach, aby ocenić efektywność algorytmów, niezbędne okazują się eksperymenty, które zostały w tej pracy inżynierskiej wykonane.