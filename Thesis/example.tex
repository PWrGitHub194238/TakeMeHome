\section{Wprowadzenie}

  Każdy rozdział rozpoczynamy od krótkiego wprowadzenia w którym omawiamy
  czym będziemy się w nim zajmować.

\subsection{Podrozdział}

To wyróżnienia twierdzeń stosujemy zdefiniowane wyżej konstrukcje
,,theorem'', ,,lemma'', ,,definition'' itp.:

\begin{definition}
  $\prod_{t\in T}A_t = \{f \in (\bigcup_{t \in T} A_t)^{T}: (\forall t \in T)(f(t) \in A_t)\}$
\end{definition}


\begin{lemma}
$e^{ia} = \cos(a) + i\sin(a)$
\end{lemma}

\begin{proof} W dowodzie skorzystamy z rozwinięć funkcji $e^x$, $\sin(x)$
oraz $\cos(x)$ w szeregi Taylora: $e^x = \sum_{n} x^n/n!$,
$\sin(x) = \sum_n (-1)^n (-1)^n x^{2n+1}/(2n+1)!$ oraz
$\cos(x) = \sum_n (-1)^n x^{2n}/(2n)!$. Korzystjąc z tych wzorów
oraz z tego, że $i^{2n} = (-1)^n$ mamy
\begin{gather*}
  e^{ia} = \sum_{n=0}^{\infty}\frac{(ia)^n}{n!} =
          \sum_{n=0}^{\infty}\frac{(ia)^{2n}}{(2n)!}
          +\sum_{n=0}^{\infty}\frac{(ia)^{2n+1}}{(2n+1)!}= \\
%
          \sum_{n=0}^{\infty}\frac{(-1)^n a^{2n}}{(2n)!}
         +i \sum_{n=0}^{\infty}\frac{(-1)^n a^{2n+1}}{(2n+1)!}= \cos(a) + i\sin(a) ~.\\
\end{gather*}
\end{proof}

\begin{corollary}
  $e^{\pi i} = -1$
\end{corollary}

Często stosowane oznaczenia definiujemy jako rozkazy. Korzystamy
ze zdefiniowanych wyżej oznaczeń na liczby rzeczywiste ($\RR$ - $\backslash$RR),
naturalne ($\NN$ - $\backslash$NN), wymierne ($\QQ$ - $\backslash$QQ) i całkowite ($\ZZ$ - $\backslash$ZZ).\\

\subsection{Podpodrozdział}

  W celu właściwego cytowania warto posłużyć się BibTex'em -
  będziecie
  mieli pewność, że bibliografia jest sformatowana poprawnie.
  Oto przykład cytowań:
  \\

  W pracy \cite{JCIRandom} pokazano, że wariancja rozkładu
  długości odcinków w protokole Chord (patrz \cite{Chord2001}) jest równa
  $\frac{1}{n^2}(\frac{1}{\ln 2} - 1 + \omega(n))$, gdzie
  $|\omega(n)| < 10^{-6}$.

  \begin{theorem}[L. Devroy \cite{DEVROYE}]
    Wartość oczekiwana długości maksymalnego odcinka wynosi
    $\frac{\ln n+\gamma}{n} + o(\frac{1}{n})$.
  \end{theorem}


\section{Opis protokołu}
\section{...}
\section{Wnioski}

Tutaj wstawiamy końcowe wnioski: to co udało się zrobić, 
to czego
nie udało się zrobić, co warto by jeszcze zrobić.