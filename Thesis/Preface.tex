\chapter{Wstęp}

Niniejsza praca porusza zagadnienie nieheurystycznych algorytmów wyszukiwania najkrótszych ścieżek w~rzeczywistych sieciach drogowych. Jej celem jest teoretyczne omówienie każdego z~przedstawianych algorytmów a~następnie, na drodze przeprowadzanych eksperymentów, jednoznaczne udzielenie odpowiedzi na pytanie, które z~nich wyróżniają się na tle pozostałych szybkością działania dla badanych sieci. W tym celu przeprowadza analizę rzeczonych algorytmów, przedstawia sposób ich działania, wskazuje mocne, słabe strony oraz poddaje seriom testów, mających na celu wykazać ich właściwości i sposób zachowania się dla zadanych sieci. W~związku z~powyższym praca jest podzielona na dwie części, w~których to kolejno skupia się na omówieniu algorytmów oraz zbadaniu ich zachowań dla wybranych grafów, reprezentujących sieci drogowe. Pierwsza z~nich~--- teoretyczna~--- zawiera opisy wszystkich implementacji, jakie później będą wykorzystane w~części drugiej, skupiając się na omówieniu ich podstawowych założeń i~idei, mając na celu zaznajomienie Czytelnika z~danym algorytmem. Każdy z~takich opisów dodatkowo wymienia (obok pseudokodu) uwagi, dotyczące szczegółów implementacyjnych danego algorytmu, sugeruje możliwości jego usprawnienia (jeżeli te są na tyle istotne, by o~nich wspominać), przeprowadza analizę najgorszego przypadku oraz zawiera ilustracje, mające na celu przedstawienie działania omawianego algorytmu na konkretnym przykładzie, ułatwiając tym samym jego zrozumienie. Pierwsza część, w~rozdziale ,,Podstawy'', dodatkowo skupia się na innych aspektach, mających wpływ na działanie oraz sposób implementacji poszczególnych algorytmów, formułując przy okazji wszystkie podstawowe definicje i~zwroty, jakie są wymagane do zrozumienia zawartości następnych rozdziałów. Część eksperymentalna zawiera wyniki przedstawiające zależności, jakie zachodzą pomiędzy poszczególnymi algorytmami a~rodzajami sieci, do jakich je zastosowano, zestawia ze sobą ich czasy działania i~wskazuje te, których efektywność jest największa. Przeprowadzone eksperymenty opierają się na implementacjach algorytmów, które zostały wykonane w~ramach tej pracy inżynierskiej i~umieszczone w~bibliotece języka \textsc{C}, o~której szerzej można się dowiedzieć w~dodatku jej poświęconej, który dodatkowo zawiera opis wszystkich narzędzi i~skryptów, które zostały w~niniejszej pracy wykorzystane.

Część implementacyjna pracy inżynierskiej została napisana w~języku \textsf{C} zgodnie ze standardami, ustalonymi przez \textit{\textsc{ISO 9899:1999}} z~wykorzystaniem środowiska programistycznego \textsc{Eclipse} w~wersji \textsc{$4.4.0$}, debuggerów \textsc{GDB} oraz \textsc{Valgrind} (w wersji \textsc{$3.10.1$}). Tekst pracy inżynierskiej został złożony w~systemie \LaTeXe~z wykorzystaniem podstawowych pakietów do reprezentacji kodu programistycznego: \textsc{listings} oraz \textsc{algorithm2e}. Do składu wszelkich ilustracji użyto aplikacji internetowej \textsc{draw.io} oraz programu \textsc{Inkscape} w~wersji \textsc{$0.48.4$}. Do wygenerowania przedstawionych w~pracy wykresów dwu- i~trójwymiarowych zastosowano programy: \textsc{Octave} (w wersji \textsc{$3.8.1$}) oraz \textsc{croppdf} (do korekty otrzymanych wykresów). Całość została skompilowana pod kontrolą systemu \textsc{Ubuntu 14.04.1 LTS}.