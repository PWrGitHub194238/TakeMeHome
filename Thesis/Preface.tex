\chapter{Przedmowa}
We wstępie przedstawiamy cel pracy, to co udało się zrobić oraz krótko omawiamy to co znajduje się w poszczególnych rozdziałach. Wstęp ma być na tyle dokładny, aby recenzent na jego podstawie mógł napisać recenzję. Wspominamy również o zastosowanych narzędziach informatycznych (np. omawiany projekt został
zrealizowany w środowisku ,,Microsoft Visual Studio 2005'', lub,że obliczenia symboliczne zostały zrealizowane za pomocą pakietu,,Mathematica'' a numeryczna za pomocą pakietu ,,Mathlab''),umieszczamy opis CD dołączonego do pracy. Uwaga: w każdej pracy dyplomowej musi być zrealizowany jakiś projekt informatyczny. \\

Pracę dzielimy na rozdziały i podrozdziały za pomocą poleceń\verb"\section{}", \verb"\subsection{}" i \verb"\subsubsection{}".Na początku każdej sekcji w kilku zdaniach opisujemy to co się w niej zawiera.\\

W całej pracy stosujemy jednolity styl: piszemy ,,pokażemy'',,,zbudujemy'', ,,udowodnimy''. Nie piszemy ,,ja pokażę'',,,udowodnię'' itp. Łatwo ten styl stosować, jeśli myślimy,że wchodzimy w dyskurs z czytelnikiem pracy.
 
Stosujemy zwięzły styl narracji. W pracy umieszczamy tylko to co jest nam potrzebne (uwaga: na obronie można być zapytanym o każdą rzecz, która znajduje się w pracy -nie piszcie więc o rzeczach,których dobrze nie rozumiecie!). Nie starajcie się w żadnym przypadku sztucznie zwiększać rozmiaru pracy (np. przez stosowanie podwójnych odstępów między liniami). 

Wszystkie cytowania należy starannie zaznaczyć i ich lokalizację umieścić w bibliografii. Dotyczy to wszystkich cytowań i zapożyczeń (również tych z internetu)